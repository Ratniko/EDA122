\newpage
\section{Discussion}
\label{S5}
Here we will discuss the up- and down-sides of the two different architectures and their modes of operation. Some general thoughts about the method used in the laboratory assignment will also be brought up.

In \secref{S4} it can easily be seen that the centralized architecture is more reliable compared to the distributed. This comes from the reliability of both the WUs and the CU being higher in the centralized architecture. It is interesting to see how much longer the centralized architecture can run before reaching the same, predetermined, level of reliability as the distributed. This comparison is made in \figref{fig:cfs_hf} and a substantial difference of $45\%$ can be seen. 

For a user of this brake-by-wire system, a longer system life time would be highly beneficial. The much longer time it takes for the centralized system to reach a reliability level of $0.92$ indicates that it would have a longer life time than the distributed. This is also supported by the $38\%$ higher MTTF for the centralized architecture, which can be seen in \tabref{tab:res}. These comparisons are made for the degraded mode of operation, but also in the full functionality mode an increase in MTTF, by $40\%$, can be observed, further indicating a longer life time for the centralized architecture.   

If looked upon from a manufacturers point of view, increasing the failure rate of the system in the centralized architecture until it's life time coincides with that of the distributed architecture, could be beneficial. In \secref{S4}, it can be seen that the failure rate of the CMs in the centralized architecture can be increased by a factor $2.2$ before the two architectures have the same reliability. And an increased failure rate, would probably lower the manufacturing and development cost for that module. Also, if the failure rate is higher, more modules would need to be replaced, and thus generate income in the form of users buying spare parts. This however, could also be true for the distributed configuration in the sense that it's WUs have a higher failure rate than the ones in the centralized configuration, and would thereby need more frequent repairs. 

When comparing the two modes of operation evaluated, degraded functionality and full functionality, it would be safe to say that the later is the only viable option for a systems continuous operation. Only as a reassurance for when the system breaks during high speed operation should the degraded mode be allowed. Otherwise the vehicle should not be used in degraded mode.

The assumptions made in this evaluation will not hold in a real system, but to get a broad picture of the design benefits without using to complicated models, the assumptions are needed. Likewise, the life-time of a vehicle is longer than the evaluated period of four years, but the reliability tendencies can easily be spotted in the results obtained from the evaluated period. Also, for any of the two evaluated architectures to fully work, additional components would be needed. Like sensors for the vehicle's change in direction, and speed. But, as these extra components are needed in both configurations, they would not affect the relation between the architectures.

The choice of implementing either of the two designs is influenced by more factors than the results of a reliability evaluation as conducted in this laboratory assignment. Other properties that is of value is for example cost, maintainability, and modifiability. The cost is naturally a major concern, and the maintenance can be a significant part of the life-time cost. It is hard for us to guess about any cost differences, as the two systems utilize the same amount of hardware. One additional point in favor for the centralized architecture is that it is probably easier to physically replace a computer module in the CU than in a WU. 

If the software should be updated with more complex routines, the CMs which derives the brake commands must possibly be upgraded. In the centralized design three CMs need to be replaced, compared to eight in the distributed design. However, the CMs replaced in the centralized design has higher processing capabilities and thereby more expensive. But as mentioned, they are probably easier to replace, and the lesser time it takes to replace the centralized CU versus the four WUs probably makes it more cost effective. 