\newpage
\section{Discussion}
\label{S5}
/{Discuss the pros and cons of the different design solutions.} /



The centralized model seems to be the better of the two regarding the results obtained in \tabref{tab:res}, it is better on all point but one. 


As the reliabilities for the designs differ, it can be interesting to see how long they can be operational until they reach a certain reliability level. The level determined was 0.92 and the hours of service before this reliability was reached for the full systems in degraded mode is presented in \figref{fig:cfs_hf}. The centralized design can operate 22\% longer than the distributed design, 31860  h versus 21960 h, respectively. If the centralized architecture do not need to have a higher reliability than the other design, a higher failure rate can be tolerated. It was determined that the failure rate for the computer modules in the CU should be increased. The failure rate could be increased to $22.21 \time 10^{-6}$ to coincide in the same point as the distributed design, as shown by the `highfail' curve in \figref{fig:cfs_hf}, 